Ein Temperatursensor liefert der Heizungssteuerung die Information, ob die Temperatur sich unter dem Fixwert befindet. Wenn das der Fall ist und die automatische Steuerung aktiv ist wird die Heizung aktiv. Unabhängig davon wird die Heizung auch aktiv, wenn dier Schalter betätigt wird. Daraus ergibt sich folgender Ablauf:
\enuerate{begin}
\item{Sensor schlägt an?}
\item{Automatische Steuerung aktiviert?}
\item{oder manuelle Steuerung aktiviert?}
\item{Heizung davon abhängig an- beziehungsweise ausschalten}
\enumerate{end}
