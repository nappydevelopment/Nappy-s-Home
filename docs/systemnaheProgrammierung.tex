\documentclass[fontsize=12pt,DIV=11,BCOR=4mm,fleqn]{scrartcl}
\KOMAoptions{footnotes=multiple}

\usepackage[hyphens]{url}
\usepackage[
	bookmarksopen,
	colorlinks=false,
	linktoc=all,
	linkcolor=red,
	citecolor=green,
	urlcolor=blue,
	pdftitle={Systemnahe Programmierung}, 
	pdfauthor={Mehmet Ali Incekara, Tom Wolske, Heiko Faller}
]{hyperref}

\usepackage[margin=3cm]{geometry}

\usepackage{tocloft}
\renewcommand{\cftsecleader}{\cftdotfill{\cftdotsep}}

\usepackage[yyyymmdd]{datetime}
\renewcommand{\dateseparator}{.}

\usepackage{enumitem}
\setlist[description]{style=nextline}

\setkomafont{section}{\Huge} 
\setkomafont{subsection}{\Large}

\usepackage[nonumberlist,acronym,toc,section]{glossaries}
\renewcommand*{\glspostdescription}{}
\makeglossaries
\include{acronym}
\include{glossar}
\glsaddall
\bibliographystyle{plaindin}

\usepackage[onehalfspacing]{setspace} 

\usepackage{fontspec}
\usepackage{microtype}
\usepackage{graphicx}
\usepackage[svgnames,table]{xcolor}
\usepackage{xcolor}
\usepackage{amsmath}
\usepackage{multicol}
\usepackage[ngerman]{babel}
\usepackage[babel,german=quotes]{csquotes}
\usepackage{eurosym}

\usepackage{listings}
\setlist[description]{style=nextline}
\definecolor{light-gray}{gray}{0.95}
\definecolor{mygreen}{rgb}{0,0.6,0}

\newcommand{\minitab}[2][c]{\begin{tabular}{#1}#2\end{tabular}}
\newcolumntype{/}{>{\global\let\currentrowstyle\relax}}
\newcolumntype{^}{>{\currentrowstyle}}
\newcommand{\rowstyle}[1]{\gdef\currentrowstyle{#1}%
  #1\ignorespaces
}

\usepackage{scrlayer-scrpage}
\automark{subsection}

\clearscrheadings
\setlength{\footheight}{36pt}
\setlength{\headheight}{28pt}
\ohead{\rightmark}
\cfoot{\\Seite \pagemark}
\pagestyle{scrheadings}

\renewcommand{\contentsname}{Inhaltsverzeichnis}
\renewcommand{\listfigurename}{Abbildungsverzeichnis}
\renewcommand{\figurename}{Abbildung}
\renewcommand{\listtablename}{Tabellenverzeichnis}
\renewcommand{\tablename}{Tabelle}

\setlength{\parindent}{0pt}

\begin{document}
	\begin{titlepage}
		\fontspec{Times New Roman}
		\begin{center}
			\includegraphics[width=8cm]{dhbw.pdf}
			
			\vfill
			\begin{singlespacing}
				{\Huge TITEL UND SO} \vspace{1.7cm}
				{\Large JSADOKAS}	\\ [0.25cm]
				{\large des Studiengangs Angewandte Informatik}	\\ [0.25cm]
				{\large an der Dualen Hochschule Baden-Württemberg Karlsruhe}	\\[0.25cm]
				{\large von} 	\\ [0.25cm]
				{\large \bfseries Heiko Faller, Mehmet Ali Incekara und Tom Wolske}	\\ [1cm]
			\end{singlespacing} 
			\vfill
		\end{center}
	\end{titlepage}
	\clearscrheadings
	
	\vspace*{2.5cm}
	\section*{Ehrenwörtliche Erklärung}
	gemäß § 5 (3) der "`Studien- und Prüfungsordnung DHBW Technik"' vom 22. September 2011. \vspace{5pt}

	Ich habe die vorliegende Arbeit mit dem Titel
	
	\begin{center}
		"` TITEL "'
	\end{center}
	
	selbstständig verfasst und keine anderen als die angegebenen Quellen und Hilfsmittel verwendet.

	\vspace{2cm}
	\makebox[7cm][l]{Montabaur, den \the\day.\the\month.\the\year}	\hfill	\\ %\makebox[5cm][c]{\includegraphics{Unterschrift_ali.png}} \\
	\rule{5cm}{0.4pt} \hfill	\rule{5cm}{0.4pt} \\
	\makebox[7cm][l]{Ort, Datum} \hfill	\makebox[5cm][c]{Mehmet Ali Incekara} 
	
	\pagebreak{}
	\clearpage
	\vspace*{2.1cm}
	\section*{Sperrvermerk}
		Die vorliegende Praxisarbeit mit dem Titel 
		
		\begin{center}
			"` TITEL "'
		\end{center}
		
		enthält vertrauliche Daten des Unternehmens 1\&1 Internet SE. \vspace{5pt}
	
		Diese Praxisarbeit darf nur vom Gutachter sowie berechtigten des Prüfungsausschusses eingesehen werden. Eine Vervielfältigung und Veröffentlichung der Praxisarbeit ist auch auszugsweise nicht erlaubt. \vspace{5pt}
	
		Dritten darf diese Arbeit nur mit der ausdrücklichen Genehmigung des Verfassers und Unternehmens zugänglich gemacht werden.
	
	\pagebreak{}
	\clearpage
	\vspace*{0.1cm}
	\tableofcontents{} \thispagestyle{empty} \addtocontents{toc}{\par}
	\addtocontents{toc}{\protect\thispagestyle{empty}}
	
	\pagebreak{}
	\clearpage
	\clearscrheadings
	\pagenumbering{Roman}
	\cfoot{\pagemark}
	\setcounter{page}{5} 
	\vspace*{2.5cm}
	\listoffigures{}\addtocontents{lof}{\par}
	\addcontentsline{toc}{section}{Abbildungsverzeichnis}
	
	\pagebreak{}
	\vspace*{2.5cm}
	\listoftables{}\addtocontents{lot}{\par}
	\addcontentsline{toc}{section}{Tabellenverzeichnis} 
	
	\pagebreak{}
	\clearscrheadings
	\pagenumbering{arabic}
	\cfoot{\pagemark}
	\setcounter{page}{1} 
	
	\section{Einleitung}
		Die Menschen wollen immer weiter vernetzt sein! Das betrifft auch das Haus in denen Sie leben. \vspace{5pt}

Smart Home dient als Oberbegriff für technische Verfahren und Systeme in Häusern, in deren Mittelpunkt eine Erhöhung von Wohn- und Lebensqualität, Sicherheit und effizienter Energienutzung steht.

\subsection{Motiviation}
	Auf die Idee ein Smarthome zu Entwickeln sind wir schnell gekommen. 
	
	Ältere Menschen in unserer Gesellschaft haben oft angst in den Urlaub zu fahren, weil sie ihr Eigenheim nicht alleine lassen wollen. Andere wollen Strom sparen um die Umwelt zu schonen. Aus diesem Grund wollen wir ein SmartHome entwickeln. Damit ältere Menschen wieder in den Urlaub gehen können und Umweltbewusste Menschen die Umwelt besser schonen können.

\subsection{Ziel der Arbeit}
	Das Ziel dieser Arbeit ist es einen Mikrocontroller zur Haussteuerung zu entwickeln. Zunächst findet die Implementierung und Ausführung als Simulation in der MCU 8051 IDE statt.

\subsection{Aufbau der Arbeit}
	Das Ziel dieser Arbeit lässt sich in Teilziele kategorisieren, um die Zielerreichung zu garantieren.
	
	Das nachfolgende Kapitel befasst sich mit den Grundlagen der Mikrocontrollerentwicklung. Anschließend wird das Konzept und die Implementierung näher erläutert. Als Abschluss folgt eine Zusammenfassung.
		
	\section{Grundlagen}
		\subsection{Der Mikrocontroller 8051}
	Der naijsdnasjdasdas
	
\subsection{MCU 8051 IDE}
qsdhuasijasd
	
	\section{Konzept}
		\subsection{Konzept und Idee}

Die Idee ist es, ein SmartHome zu entwickeln, welches auf einem 8051 Microcontroller als Grundlage aufbaut. Dabei sollen drei Hauptfunktionen eines modernen SmartHomes über diesen gesteuert werden.

\subsubsection{Heizung}
Die erste Funktion ist eine Temperatursteuerung für die Fußbodenheizung. Dabei soll die Fußbodenheizung in einem Haus so gesteuert werden, dass diese automatisch eingeschaltet wird, sobald die Temperatur unter einen festen Wert fällt. Außerdem soll der Nutzer mit Hilfe eines Schalters festlegen können, ob die Heizung dauerhaft an oder in einen automatischen Modus eingestellt ist. Der Temperatursensor ist zusätzlich abschaltbar, damit die Heizung ausgeschaltet werden kann.

Wird der Heizungsschalter auf An gestellt, läuft die Heizung unabhängig von der Temperatur die ganze Zeit. Wird dieser auf den Automodus gestellt, kommt es auf den Temperatursensor an, da dieser unabhängig von der Heizung ein bzw. ausgeschaltet werden kann. Ist dieser eingeschaltet, schaltet sich die Heizung an, sobald die Temperatur unter einen bestimmten Wert fällt.
Aus diesen Anforderungen ergibt sich eine Tabelle aus der alle möglichen Eingaben und Ausgabemöglichkeiten ausgelesen werden können. Aus diesen geht ein Schaltplan hervor, welcher für die Programmierung des 8051 Microcontrollers genutzt wird.
Die Tabelle sieht wie folgt aus:

\begin{table}[htbp]
\centering
\caption{Schaltplan Heizungssteuerung}
\label{my-label}
\begin{tabular}{|l|l|l|l|}
\hline
\multicolumn{1}{|c|}{\textbf{Heizungsschalter}} & \multicolumn{1}{c|}{\textbf{Temperatursensor Schalter}} & \multicolumn{1}{c|}{\textbf{Temperatur < Wert}} & \multicolumn{1}{c|}{\textbf{Heizung An?}} \\ \hline
 0 & 0 & 0 & 0 \\ \hline
 0 & 0 & 1 & 0 \\ \hline
 0 & 1 & 0 & 0 \\ \hline
 0 & 1 & 1 & 1 \\ \hline
 1 & 0 & 0 & 1 \\ \hline
 1 & 0 & 1 & 1 \\ \hline
 1 & 1 & 0 & 1 \\ \hline
 1 & 1 & 1 & 1 \\ \hline
\end{tabular}
\end{table}

Für diese Steuerung wird die Schaltung nach diesem Schaltplan programmiert:

\begin{figure}[htbp] 
  \centering
     \includegraphics[width=0.7\textwidth]{Heizungsschaltung.png}
  \caption{Schaltung für die Heizung}
  \label{fig:Bild1}
\end{figure}

Dabei kommen die Eingaben aus dem Port 2.0 bis 2.2 und die Ausgabe ist auf dem Port 3.4.

\subsubsection{Licht}
Die zweite Funktion ist eine Lichtsteuerung für das Licht. Der Nutzer soll die Möglichkeit haben das Licht mit Hilfe eines Bewegungssensors ein beziehungsweise aus zu schalten. Dieser Sensor ist zusätzlich an einen Schalter angeschlossen, um diesen An bzw. Aus zu schalten.
 
Aus diesen Anforderungen ergibt sich eine Tabelle aus der alle möglichen Eingaben und Ausgabemöglichkeiten ausgelesen werden können. Aus diesen geht ein Schaltplan hervor, welcher für die Programmierung des 8051 genutzt wird.

\begin{table}[htbp]
\centering
\caption{Schaltplan Lichtsteuerung}
\label{my-label}
\begin{tabular}{|l|l|l|}
\hline
\multicolumn{1}{|c|}{\textbf{Bewegungssensor}} & \multicolumn{1}{c|}{\textbf{Schalter}} & \multicolumn{1}{c|}{\textbf{Licht an?}} \\ \hline
 0 & 0 & 0 \\ \hline
 0 & 1 & 0 \\ \hline
 1 & 0 & 0 \\ \hline
 1 & 1 & 1 \\ \hline
\end{tabular}
\end{table}

Für diese Steuerung wird die Schaltung nach diesem Schaltplan programmiert:

\begin{figure}[htbp] 
  \centering
     \includegraphics[width=0.7\textwidth]{Lichtschaltung.png}
  \caption{Schaltung für das Licht}
  \label{fig:Bild2}
\end{figure}

Dabei kommen die Eingaben aus den Ports P1.0 und 1.1 und die Ausgabe ist auf dem Port 3.5.

\subsubsection{Rollläden}
Der dritte Teil der SmartHome Steuerung ist eine Steuerung für Rollläden. Diese werden in zwei Modi betrieben. Im automatischen Modus sollen sie sich automatisch öffnen, sobald es draußen hell, beziehungsweise schließen, sobald es draußen dunkel wird. Es wird ein Helligkeitssensor benötigt, der ein bit setzt, sobald es hell ist und umgekehrt. Der zweite Modus ist eine manuelle Steuerung. Für jeden Rollladen gibt es je zwei Schalter, die für Rollladen Hoch, beziehungsweise für Rollladen runter, stehen. Beide Modi werden durch Sensoren unterstützt, die jeweils anzeigen, ob ein Rollladen geschlossen oder ganz geöffnet ist. 
Aus diesen Anforderungen gehen wie zuvor auch diese Tabellen für die beiden Modi hervor:

\begin{table}[]
\centering
\caption{Rolladensteuerung manuell}
\label{my-label}
\begin{tabular}{|l|l|l|l|l|l|}
\hline
\multicolumn{1}{|c|}{\textbf{SR Oben}} & \multicolumn{1}{c|}{\textbf{SR Unten}} & \multicolumn{1}{c|}{\textbf{SR Hoch}} & \multicolumn{1}{c|}{\textbf{SR Runter}} & \multicolumn{1}{c|}{\textbf{MR Hoch}} & \multicolumn{1}{c|}{\textbf{MR Runter}} \\ \hline
 0 & 0 & 0 & 0 & 0 & 0 \\ \hline
 0 & 0 & 0 & 1 & 0 & 1 \\ \hline
 0 & 0 & 1 & 0 & 1 & 0 \\ \hline
 0 & 0 & 1 & 1 & 0 & 0 \\ \hline
 0 & 1 & 0 & 0 & 0 & 0 \\ \hline
 0 & 1 & 0 & 1 & 0 & 0 \\ \hline
 0 & 1 & 1 & 0 & 1 & 0 \\ \hline
 0 & 1 & 1 & 1 & 0 & 0 \\ \hline
 1 & 0 & 0 & 0 & 0 & 0 \\ \hline
 1 & 0 & 0 & 1 & 0 & 1 \\ \hline
 1 & 0 & 1 & 0 & 0 & 0 \\ \hline
 1 & 0 & 1 & 1 & 0 & 0 \\ \hline
 1 & 1 & 0 & 0 & 0 & 0 \\ \hline
 1 & 1 & 0 & 1 & 0 & 0 \\ \hline
 1 & 1 & 1 & 0 & 0 & 0 \\ \hline
 1 & 1 & 1 & 1 & 0 & 0 \\ \hline
\end{tabular}
\end{table}


\begin{table}[]
\centering
\caption{Rolladensteuerung automatisch}
\label{my-label}
\begin{tabular}{|l|l|l|l|l|}
\hline
\multicolumn{1}{|c|}{\textbf{SR Oben}} & \multicolumn{1}{c|}{\textbf{SR Unten}} & \multicolumn{1}{c|}{\textbf{Helligkeitssensor}} & \multicolumn{1}{c|}{\textbf{MR Hoch}} & \multicolumn{1}{c|}{\textbf{MR Runter}} \\ \hline
0 & 0 & 0 & 0 & 1 \\ \hline
0 & 0 & 1 & 1 & 0 \\ \hline
0 & 1 & 0 & 0 & 0 \\ \hline
0 & 1 & 1 & 1 & 0 \\ \hline
1 & 0 & 0 & 0 & 1 \\ \hline
1 & 0 & 1 & 0 & 0 \\ \hline
1 & 1 & 0 & 0 & 0 \\ \hline
1 & 1 & 1 & 0 & 0 \\ \hline
\end{tabular}
\end{table}
Die Signale aus Sensoren und Schalter sollen über P0, sowie über P2.6 und P2.7 eingegeben werden. Die Ausgabe an die Motoren erfolgt über P3.0 - P3.3

	
	\section{Umsetzung}
		\subsection{Rolladen-Steuerung}
	Die Schwierigkeit bei der Entwicklung der Rollladenschaltung war, dass eine Menge Port-Bits benötigt werden, weshalb wir uns dazu entschieden haben, ein Modell zur Steuerung von zwei Rollläden zu wählen. Als erstes wird das Automodus-Bit abgefragt, um zu entscheiden, ob der Automodus eingesetzt wird oder die Röllläden manuell gesteuert werden.
\subsubsection{Manuelle Steuerung}
Im manuellen Modus wird bei den Röllläden abwechselnd überprüft, ob sie hoch oder runter fahren sollen. Dazu wird mittels Polling jeweils abgefragt, wie die Schalterpositionen sind und ob der zu überprüffende Rollladen oben oder unten ist. Dementsprechend ist die Rollladenposition 
\subsubsection{Audomodus}


\subsection{Licht-Steuerung}
	Bei der Lichtsteuerung werden einfach die Eingänge verundet und das Ergebnis an das Licht ausgegeben. Der Ablauf sieht also folgendermaßen aus.
\enuerate{begin}
\item{Automatische Lichtsteuerung aktiviert?}
\item{Sensor schlögt an?}
\item{Licht davon abhängig an- beziehungsweise ausschalten}
\enumerate{end}


\subsection{Heizung-Steuerung}
	Ein Temperatursensot liefert der Heizungssteuerung die Information, ob die Temperatur sich unter dem Fixwert befindet. Wenn das der Fall ist und die automatische Steuerung aktiv ist wird die Heizung aktiv. Unabhängig davon wird die Heizung auch aktiv, wenn dier Schalter betätigt wird. Daraus ergibt sich folgender Ablauf:
\enuerate{begin}
\item{Sensor schlägt an?}
\item{Automatische Steuerung aktiviert?}
\item{oder manuelle Steuerung aktiviert?}
\item{Heizung davon abhängig an- beziehungsweise ausschalten}
\enumerate{end}

		
	\section{Zusammenfassung}
		Ziel war die beispielhafte Entwicklung einer Smarthome Haussteuerung auf dem Mikrocontroller 8051. Dabei wurden drei verschiedene Funktionen umgesetzt:

\begin{itemize}
	\item {Eine Heizungssteuerung}
	\item {Eine Lichtsteuerung}
	\item {Eine Rolladensteueung}
\end{itemize}

Diese Funktionen wurden in dem Emulator MCU 8051 programmiert und beispielhaft ausgeführt.


Abschließend kann gesagt werden, dass diese Funktionen funktionsfähig sind, der Funktionsumfang jedoch noch durch weitere nützliche Automatisierungen ergänzt werden kann. Dazu wird jedoch weitere Hardware benötigt und die Anzahl der Ports am 8051 Mikrokontroller sind begrenzt.


		
\end{document}